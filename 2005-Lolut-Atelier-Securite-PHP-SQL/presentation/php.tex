\section{PHP}

\subsection{Pr�sentation de PHP}

\begin{frame}
  \frametitle{Pr�sentation de PHP}

  \begin{columns}
    \begin{column}{1.0\textwidth}<1->
      \begin{itemize}
      \item<1-> PHP est un langage interpr�t� �crit par \textit{Rasmus Lerdorf} en 1994.
      \item<2-> Il est tr�s utilis� aujourd'hui dans les sites Internet (25~millions de site en 2005).
      \item<3-> Il est appr�ci� pour sa facilit� d'utilisation, la grande quantit� des biblioth�ques incluses,
      son aspect "logiciel libre", et enfin sa grande communaut� d'utilisateurs (documentation).
      \end{itemize}
    \end{column}
  \end{columns}
\end{frame}

\subsection{Faiblesses de PHP}

\begin{frame}
  \frametitle{Failles PHP}

  \begin{columns}
    \begin{column}{1.0\textwidth}<1->
      \begin{itemize}
      \item<1-> PHP est souvent le premier langage appris par les concepteurs de site Internet.
      \item<2-> La s�curit� est souvent n�glig�e par manque d'exp�rience, ou simplement par
      manque d'int�r�t.
      \item<3-> Les failles de s�curit� sont plus ou moins graves, mais trop souvent
      on peut avec un peu d'exp�rience p�n�trer dans un site sans grande difficult�.
      \item<4-> Vieil adage : � 90\% des erreurs sont situ�es entre le clavier
      et l'�cran � :-)
      \end{itemize}
    \end{column}
  \end{columns}
\end{frame}

\subsection{Exemples de failles}

\begin{frame}
  \frametitle{Exemples de failles}

  \begin{columns}
    \begin{column}{1.0\textwidth}<1->
      \begin{itemize}
      \item<1-> Formulaire v�rifiant de mani�re incompl�te les entr�es des utilisateurs. 
      \item<2-> Outil de t�l�chargement permettant de t�l�charger n'importe quel fichier.
      \item<3-> Injection de SQL (d�taill�e plus tard).
      \item<4-> etc. (la liste est longue)
      \end{itemize}
    \end{column}
  \end{columns}
\end{frame}

\subsection{Limiter la casse}

\begin{frame}
  \frametitle{Limiter la casse}

  \begin{columns}
    \begin{column}{1.0\textwidth}<1->
      \begin{itemize}
      \item<1-> Supposer que ce qui vient de l'ext�rieur provient d'une
      personne qui cherche � nous nuire.
      \item<2-> Ne pas expliciter les cas � proscrire, mais plut�t les cas
      � autoriser.
      \item<3-> Concevoir l'architecture du syst�me pour isoler les parties
      sensibles.
      \end{itemize}
    \end{column}
  \end{columns}
\end{frame}
