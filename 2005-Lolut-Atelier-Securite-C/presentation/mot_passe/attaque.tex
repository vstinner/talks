\section{Diff�rentes attaques}

\subsection{Attaque par dictionnaire}

\begin{frame}
  \frametitle{Attaque par dictionnaire}

  \begin{columns}
    \begin{column}{1.0\textwidth}<1->
      \begin{itemize}
      \item<1-> Pour attaquer un syst�me d'authentification, on peut tester
      les mots de passe les plus courant. On utilise pour cela des
      dictionnaires.
      \item<2-> Le programme d'attaque utilise les astuces courantes comme
      ajouter un nombre � la fin d'un mot (jojo42) ou remplacer une lettre
      par un chiffre (h4k3r).
      \item<3-> La solution pour �viter ces attaques est de limiter le
      nombre d'essai d'authentification et/ou ajouter un d�lai apr�s un
      �chec.
      \end{itemize}
    \end{column}
  \end{columns}
\end{frame}

\subsection{Attaque par force brute}

\begin{frame}
  \frametitle{Attaque par force brute}

  \begin{columns}
    \begin{column}{0.9\textwidth}<1->
      \begin{itemize}
      \item<1-> Parfois, un attaque par force brute est envisageable. Elle
      consiste � tester tous les mots de passe possibles.
      \item<2-> On teste par exemple les mots de passe entre 1 et 8
      caract�res compos�s de lettres ou de chiffre.
      \item<3-> Mais c'est le cas de dernier recours, le social engineering
      est bien plus efficace !
      \end{itemize}
    \end{column}
  \end{columns}
\end{frame}

\subsection{Attaque de hash}

\begin{frame}
  \frametitle{Attaque de hash}

  \begin{columns}
    \begin{column}{0.9\textwidth}<1->
      \begin{itemize}
      \item<1-> Une fonction de hashage est tr�s difficilement r�versible,
      mais c'est possible. Le principe consiste � essayer un mot de passe, calculer
      son hash et le comparer au hash qu'on veut "casser".
      \item<2-> Le logiciel RainbowCrack pr�calcule des hashs et poss�de
      un algorithme d'acc�s tr�s rapide. Il produit des fichiers d'une
      taille allant jusqu'� 64 Go. 
      \end{itemize}
    \end{column}
  \end{columns}
\end{frame}

\subsection{Social engineering}

\begin{frame}
  \frametitle{Social engineering}
  \framesubtitle{Because human stupidity has no limit}

  \begin{columns}
    \begin{column}{1.0\textwidth}<1->
      \begin{itemize}
      \item<1-> Le social engineeing
      est une m�thode permettant de soustraire une information
      ou un bien en usant de l'ignorance et la na�vet� de sa victime.
      
      \item<2-> Cette technique peut se pratiquer par t�l�phone (moyen le plus
      simple et rapide), par email, par lettre, ou par contact direct.
      
      \item<3-> La m�thode la plus connue est l'hame�onage (phishing en
      anglais) qui consiste � envoyer un email � sa victime lui demandant
      d'aller sur un faux site pour revalider son mot de passe.

      \item<4-> La meilleure solution reste la parano�a
      \end{itemize}
    \end{column}
  \end{columns}
\end{frame}
