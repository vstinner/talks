\documentclass[handout]{beamer}

\mode<presentation>
{
  \usetheme{Warsaw}
  \usefonttheme{structurebold}
}
\setbeamercovered{dynamic}
\usepackage[utf8]{inputenc}
\usepackage{times}
\usepackage{graphicx}
\usepackage[T1]{fontenc}

\title[Fonctionnalités sexy de Python]{Python bling bling}

\author{Victor Stinner}

\date{Pycon FR, Paris, mai 2009}

\beamerdefaultoverlayspecification{<+->}

\begin{document}

\begin{frame}
    \titlepage
\end{frame}

\section{Syntaxe}

\subsection{Opérateurs}
\frame
{
\begin{columns}[c]
\column{0.4\textwidth}
    \includegraphics[height=0.9\textheight]{blingbling}
\column{0.6\textwidth}
    \begin{itemize}
    \item 1 < x <= 10
    \item 0 <= x < y <= 20
    \item a, b = b, a + b \# Fibonacci
    \item color = "red" if load > 0.8 else "green"
    \end{itemize}
\end{columns}
}

\subsection{Slices}
\begin{frame}[fragile]
\begin{columns}[c]
\column{0.4\textwidth}
    \includegraphics[height=0.9\textheight]{blingbling}
\column{0.6\textwidth}
\begin{verbatim}
>>> carres = [1, 4, 9, 16, 25]
>>> carres[1:-1]
[4, 9, 16]
>>> carres[::2]
[1, 9, 25]
>>> carres[::-1]
[25, 16, 9, 4, 1]
\end{verbatim}
\end{columns}
\end{frame}

\section{Itérateurs}

\subsection{Utilisation}
\begin{frame}
\begin{columns}[c]
\column{0.6\textwidth}
    \begin{itemize}
    \item str, unicode, list, tuple, dict, set, <fichier>, ...
    \item dict.iterkeys(), dict.itervalues(), dict.iteritems()
    \item iter(object) ou "for element in object: ..."
    \item Votre objet : méthode \_\_iter\_\_()
    \end{itemize}
\column{0.4\textwidth}
    \includegraphics[width=\textwidth]{holygrail}
\end{columns}
\end{frame}

\subsection{Outils}
\begin{frame}[fragile]
    \begin{columns}[c]
    \column{0.6\textwidth}
        \begin{verbatim}
>>> list(enumerate("abc"))
[(0, 'a'), (1, 'b'), (2, 'c')]
>>> list(zip("ab", "XY"))
[('a', 'X'), ('b', 'Y')]
>>> gen = itertools.count()
>>> gen.next(), gen.next()
(0, 1)
        \end{verbatim}
    \column{0.4\textwidth}
        \includegraphics[width=\textwidth]{holygrail}
    \end{columns}
\end{frame}

\section{Générateurs}

\subsection{Syntaxe}
\begin{frame}
    \begin{columns}[c]
    \column{0.4\textwidth}
        \includegraphics[width=\textwidth]{chapman}
    \column{0.6\textwidth}
        \begin{itemize}
        \item (item[2] for item in items)
        \item imap(operator.itemgetter(2), items)
        \item (item.name for item in items)
        \item imap(operator.attrgetter("name"), items)
        \end{itemize}
    \end{columns}
\end{frame}

\subsection{Exemple}
\begin{frame}[fragile]
    \begin{columns}[c]
    \column{0.4\textwidth}
        \includegraphics[width=\textwidth]{chapman}
    \column{0.6\textwidth}
        \begin{verbatim}
def fibonacci():
  u, v = 1, 0
  while 1:
    yield u
    u, v = v, u + v
        \end{verbatim}
    \end{columns}
\end{frame}

\subsection{Intérêt}
\begin{frame}
    \begin{columns}[c]
    \column{0.4\textwidth}
        \includegraphics[width=\textwidth]{chapman}
    \column{0.6\textwidth}
        \begin{itemize}
        \item Simple
        \item Rapide
        \item Générique
        \item Suite infinie
        \end{itemize}
    \end{columns}
\end{frame}

\subsection{Types simples}
\begin{frame}
    \begin{columns}[c]
    \column{0.4\textwidth}
        \includegraphics[width=\textwidth]{chapman}
    \column{0.6\textwidth}
        \begin{itemize}
        \item list : [(3 + x**2) for x in xrange(10, 1000)]
        \item tuple : tuple((3 + x**2) for x in xrange(10, 1000))
        \item dict : dict((cle, valeur) for cle, valeur in elements)
        \item ...
        \end{itemize}
    \end{columns}
\end{frame}

\section{with}

\subsection{Avant}
\begin{frame}[fragile]
    \begin{columns}[c]
    \column{0.6\textwidth}
        \begin{verbatim}
f = open('text', 'w')
f.write('hello!')
f.close()
        \end{verbatim}
    \column{0.4\textwidth}
        \includegraphics[width=\textwidth]{foot}
    \end{columns}
\end{frame}

\subsection{Après}
\begin{frame}[fragile]
    \begin{columns}[c]
    \column{0.6\textwidth}
        \begin{verbatim}
# Python 2.5
from __future__ \
 import with_statement
with open('text', 'w') as f:
    f.write('hello!')
        \end{verbatim}

        Fichier, verrou, vos propres objets, etc.
    \column{0.4\textwidth}
        \includegraphics[width=\textwidth]{foot}
    \end{columns}
\end{frame}

\section{Décorateur}

\subsection{Existants}
\begin{frame}
    \begin{itemize}
    \item @classmethod
    \item @staticmethod
    \item @property
    \end{itemize}
\end{frame}

\subsection{Example}
\begin{frame}[fragile]
\begin{verbatim}
def deprecated(comment=None):
    def _deprecated(func):
        def newFunc(*args, **kwargs):
            message = "Call deprecated function %s" % func.__name__
            if comment:
                message += ": " + comment
            warn(message, category=DeprecationWarning, stacklevel=2)
            return func(*args, **kwargs)
        newFunc.__name__ = func.__name__
        newFunc.__doc__ = func.__doc__
        return newFunc
    return _deprecated
\end{verbatim}
\end{frame}

\subsection{Utilisation}
\begin{frame}[fragile]
\begin{verbatim}
@deprecated
def initialize_random():
    ...

@deprecated("Use TimedeltaWin64 field type")
def durationWin64(field):
    ...
\end{verbatim}
\end{frame}

\section{Bonus}

\subsection{set}
\begin{frame}[fragile]
    \begin{columns}[c]
    \column{0.4\textwidth}
        \includegraphics[height=0.8\textheight]{pythonsexy}
    \column{0.6\textwidth}
        \begin{verbatim}
>>> "a" in set("abc")
True
>>> set("abc") | set("ad")
set(['a', 'c', 'b', 'd'])
>>> set("abc") - set("ad")
set(['c', 'b'])
        \end{verbatim}
    \end{columns}
\end{frame}

\subsection{Modules standards}
\begin{frame}
    \begin{columns}[c]
    \column{0.4\textwidth}
        \includegraphics[height=0.8\textheight]{pythonsexy}
    \column{0.6\textwidth}
        \begin{itemize}
        \item datetime : date et heure
        \item logging : loguer du texte
        \item optparse : arguments sur la ligne de commande
        \item pprint.pprint() : \textit{pretty print}
        \end{itemize}
    \end{columns}
\end{frame}

\subsection{Modules externes}
\begin{frame}
    \begin{columns}[c]
    \column{0.4\textwidth}
        \includegraphics[height=0.8\textheight]{pythonsexy}
    \column{0.6\textwidth}
        \begin{itemize}
        \item BeautilfulSoup : parser du HTML non conforme
        \item PIL : Python Imaging Library
        \end{itemize}
    \end{columns}
\end{frame}

\subsection{Questions ?}
\begin{frame}
    \begin{center}
        \includegraphics[height=0.9\textheight]{pythonsexy}
    \end{center}
\end{frame}

\subsection{Sources}
\begin{frame}
    \begin{itemize}
    \item http://upload.wikimedia.org/wikipedia/en/9/90/Chapman\_as\_Brian.jpg
    \item http://www.flickr.com/photos/chris-parry/3007001331/
    \item http://www.flickr.com/photos/barrykidd/1763691970/
    \item http://stackoverflow.com/questions/101268
    \item http://python.net/~goodger/projects/pycon/2007/idiomatic/
    \item Trucs et astuces, Victor Stinner, Hors Série Linux Magazine 40 (janvier/février 2009)
    \end{itemize}
\end{frame}

\end{document}
